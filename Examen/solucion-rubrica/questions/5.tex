%!TEX root = ../main/main.tex

Un Message Authentication Code es una función para autentificar mensajes en base a una llave simétrica. Esta función toma una llave $k$ y un mensaje $m$ para generar un tag $t$. Intuitivamente, esperamos que una persona con la llave $k$ pueda verificar que el tag $t$ es válido para el mensaje $m$, y que alguien sin acceso a dicha llave no pueda autentificar mensajes que no se han autentificado antes. Para formalizar esta noción, en clases definimos un \emph{juego} que constaba de cinco pasos.
\begin{enumerate}
  \item El verificador genera una llave $k$ al azar.
  \item El adversario envía un mensaje $m$ al verificador.
  \item $\ldots$
  \item Los pasos 2 y 3 se repiten tantas veces como quiera el adversario.
  \item $\ldots$
\end{enumerate}
Escriba los dos pasos faltantes, y luego explique cuándo decimos que el adversario gana el juego.

\textbf{Solución:} Los pasos faltantes son:
\begin{itemize}
\item[(c)] El verificador responde con el tag $t$ generado con la llave $k$ y el mensaje $m$.
\item[(e)] El adversario envía un mensaje $m_0$ junto a un tag $t_0$, tal que $m_0$ es distinto de todos los mensajes enviados en el paso (b).
\end{itemize}
El adversario gana cuando $t_0$ es un tag válido para el mensaje $m_0$ dada la llave $k$.

\textbf{Corrección:} Se considerará correcta una respuesta que sea equivalente a lo anterior. En particular, la condición de que $m_0$ sea distinto a todos los mensajes enviados anteriormente podría considerarse como parte de lo necesario para que el adversario gane el juego.
