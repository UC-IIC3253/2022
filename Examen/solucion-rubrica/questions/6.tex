%!TEX root = ../main/main.tex

Ciertos documentos son aceptados por instituciones cuando están verificados por un notario y el notario asegura que el documento se puede utilizar en dicha entidad. A modo de ejemplo, la Universidad Católica aceptará una fotocopia de un título profesional siempre que un notario firme un mensaje que diga ``Esta fotocopia de su título es válida, y se puede utilizar en la Universidad Católica''.

Suponga que se quiere digitalizar lo explicado arriba bajo los siguientes supuestos:
\begin{itemize}
  \item Existe un solo notario que tiene una llave secreta y una llave pública.
  \item Todo el mundo conoce la llave pública del notario.
  \item Cada entidad $E$ tiene un string identificador $\text{ID}_E$ conocido por todos. Todos estos identificadores tienen el mismo largo.
\end{itemize}

Ahora, cuando una entidad $E$ pida a un cliente un documento firmado y autorizado por el notario, se usará el siguiente protocolo.

\begin{enumerate}
  \item El cliente genera el documento $d$ y encripta $\text{ID}_E || d$ con la llave pública del notario (suponemos que el documento es también un string). El texto cifrado resultante, que llamaremos $c$, es enviado al notario.
  \item El notario verifica que el documento es válido y se puede utilizar en la entidad correspondiente. De no ser así se termina el protocolo.
  \item El notario genera una firma $f$ para $c$ con su llave privada y le envía $f$ al cliente.
  \item Finalmente, el cliente envía $(c,f)$ a la entidad $E$.
\end{enumerate}

Para tener esta pregunta correcta deberá responder correctamente a las siguientes tres preguntas: (1) ¿Por qué el protocolo no satisface lo esperado? (2) ¿Cómo podemos modificar el último paso para arreglarlo? (3) ¿Qué complejidad innecesaria tiene el protocolo?

\textbf{Solución:}
\begin{itemize}
\item[(1)] Porque la entidad $E$ al recibir $(c, f)$ no puede verificar que esta firma corresponde al documento que se necesita, puesto que $c$ sólo se puede decriptar con la llave secreta del notario.
\item[(2)] Basta con enviar $(d,f)$ en lugar de $(c,f)$. Con esto, la entidad $E$ podría primero encriptar $\text{ID}_E || d$ con la llave pública del notario para obtener $c$, y luego verificar que $f$ es una firma válida para $c$.
\item[(3)] Para el objetivo que se busca no es necesario encriptar el documento. El notario podría directamente firmar $\text{ID}_E || d$.
\end{itemize}

\textbf{Corrección:} Se considerará correcta una solución que tenga una respuesta a la pregunta (1) que indique que $E$ no puede validar el documento porque no puede decriptar $c$. Las respuestas a las otras dos preguntas deben simplemente satisfacer los requerimientos del protocolo, no es necesario que sean equivalentes a lo que se menciona arriba. Obviamente, la respuesta a (2) sólo debe modificar el paso (d).
